\begin{abstract}
Todays storage capabilities facilitate the accessibility and long term archival of increasingly large data sets usually refered to as "Big Data". Tree-structured hierarchical data is very common, for instance phylogenetic trees, filesystem data, syntax trees and often times organizational structures. Analysts often face the problem of gathering information through comparison of multiple trees. However large amounts of data lead to an information overload. Visual analytic tools aid analysts by combining visual clues and analytical reasoning. Visual representations are ideal as they tend to stress human strength which are great at interpreting visualizations.

We therefore propose a prototype for comparing tree-structures which either evolve through time or usually share large node-sets. Our backend Treetank is a tree-storage system designed to persist several revisions of a tree-structure efficiently. Differend types of similarity measures are implemented adhering to the well known tree-to-tree edit problem. The Fast Matching Simple Editscript-algorithm published in a paper by Chawathe is implemented to support the comparison of ID-less trees and to store these differences as a new revision in Treetank. Based on unique stable node-IDs through time we are able to use a novel ID-based diff-algorithm to build an aggregated tree-structure formed by comparing different revisions.

The aggregated tree-structure is input to several interactive visualizations. A novel Sunburst-layout facilitates the comparison between two revisions and besides providing several other visualization options supports zooming as well as drilling down into the tree by selecting a new root node. Besides inserted, deleted and updated nodes moved- and replaced-nodes are optionally depicted to improve expressiveness. Hierarchical edge bundles are used to visualize moves. Furthermore several filtering-techniques are available to compare even very large tree-structures up to many hundred thousand or even millions of nodes. Small multiple variants of the Sunburst-layout aid the comparison between multiple trees. A text view furthermore depicts a serialization of the aggregated tree-structure in form of syntax highlighted XML. To support large trees only visible content is serialized plus an additional overhead to support scrolling which appends lines on demand. The views are synchronized.

A short evaluation based on a few criterias and a study of three application scenarios as well as performance evaluations proves the applicability of our approach. In comparison to other approaches to the best of our knowledge besides TreeJuxtaposer our approach is the only one which facilitates the comparison of large trees. Furthermore we support the comparsion of generic rooted, labeled trees whereas other approaches are specifically designed for certain domains and often times either the comparison is based on IDs or the algorithm is not usable for other domains. Thus our approach clearly surpasses most other approaches in terms of generability and scalability due to our database driven approach which allows for a fast ID-based difference algorithm optionally using hashes for filtering changed subtrees.
\end{abstract}
